\documentclass[11pt]{article}
\usepackage{graphicx}	
\begin{document}

\begin{center}
{\bf {\large
Proof for the Stationary Circle  (Monge-Darboux?)}}
\end{center}

Sergei Tabachnikov gave us an elegant argument.  It not only proves that we have a circle but also gives its radius: \\

\noindent  Observation xxx.   The radius of the Monge-Darboux circle  is the inverse
of the Joachimsthal integral. \\ 

Let the outer ellipse be given by $(Ax,x)=1$, and let the unit vector from $x$ to $y$ be $v$ .  Recall that Joachimsthal integral of the billiard map can be cast as $(Ax,v)=c$, and one has:\\

i)  $(Ax,v) = -(Ay,v)$   (which is one half of the proof that it is an integral, and  in fact can be proved by a quick calculation). Hence 
$$ (Ax+Ay,v)=0.$$
ii) The equation of the tangent line at point  $-y$  (here is where the reflection is taken)  is ($Ay,z)=-1$ , and that of the tangent line at point $x$ is $(Ax,z)=1$
 (this is because $Ax$ is a normal vector at $x$, and likewise at $y$.) \\ \\
iii)  Hence we have, for the intersection point $z$:  
$$ (Ax+Ay,z)=0. $$
It follows that $z$ is proportional to $v$.  See the figure just for a visual confirmation.
Finally,  since $(Ax,v)=c$ and ($Ax,z)=1$, we have
$$ z = v/c .$$
That is, $z$ lies on the circle of radius $1/c$.\\

{\small
%one now knows that z sweeps the red circle.
\noindent  Remark.  The green line and  the green ellipse are not used in the proof.  They  illustrate a duality.
Recall  the conceptual level, a quadratic form defines a linear isomorphism between a space and its dual space and, in dimension 2, 
 provides an identification between points and lines, a projective duality.
Thus, by definition, the dual line to point z connects the two tangency points of the tangent line to the  conic
being considered. % from z.
In other words, the line 
$ L = L(z) $  passing through $ x$ and  $-y$ is, by definition, the dual of $ z $with respect to the outer blue ellipse.
 (the billiard). %We are denoting the dependency  of   $x$ and $y$ on $z $ as x(z) and \pm y(z) and 
 %L(z) the corresponding green line.
 Consider the 1 parameter family of green lines $L = L(z)$ as z sweeps the circle. 
 The caustic of this family is an ellipse, which lies intermediary between the blue ones. It is not  confocal to the blue ones. }
% (one the one hand, I checked it numerically, and on the other, why should it be?)

 
 
  \centering
 \includegraphics[scale=0.3]{circle.png}

\end{document}