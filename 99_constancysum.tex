\documentclass[11pt]{article}
\usepackage{graphicx}	
\begin{document}

\begin{center}
{\bf {\large
Proof that the sum of cosines is an integral}}
\end{center}

Sergei Tabachnikov has a proof (joint with Arseniy Akopian and Richard Schwartz) and kindly sent us the sketch.  The idea is to use the
``meromorphic technique''.  We look forward for the details when their paper is ready.   Using  the Joachimsthal integral, recall that the cosine
of an internal angle is, up to an additive constant and signs given by  $1/|Ax|^2$.  

Follows Sergei's message:

``The idea is to complexify and to analyze the poles of the sum of  $1/|Ax|^2 $ over the orbit as a function on the elliptic curve covering the conic.
It turns out that the poles cancel each other, so one has a meromorphic function without poles, hence a constant.

In fact, one can obtain [(n-1)/2] conserved quantities by considering star-shaped associated polygons that are also billiard orbits (in confocal ellipses). 
In plain terms, these are sums of cosines of the sums of two consecutive angles, three consecutive angles, etc. of the polygon that is an n-periodic orbit. 
The first time a new invariant shows up is when n=5.''

In some more details:

``We are given two conics,$ C_1$ and $C_2$, and we assume everything is over the field of complex numbers.
The main character if the set of flags $E$ consisting of a point on $C_1$ and a line through this point tangent to $C_2$. This is an algebraic curve of genus 1, topologically, a torus.
 On $E$, one has two involutions, the one that interchanges two points of $C_1$ on the same line and the one that swaps two lines through the same points of $C_1$. The composition is the Poncelet map of $E$, and it is a parallel translation of the torus.
 
Now consider the function $F=\sum 1/|Ax|^2$. One can lift it from $C_1$ to $E$, and one wants to show that it is constant. For this, it suffices to show that it has no poles. So when does $(Ax,Ax)$ vanishes.   {\it One calculates} that these are four points of $C_1$, the tangency points of the four tangent lines that are shared by the family of confocal conics, in particular, $C_1$ and $C_2$ (in the real picture we do not  see these tangents).

Now consider a fragment of a periodic orbit on E that involves one of these points:
$$ \cdots (D,c)  \to  (C,b) \to  (B,a)  \to (A,a) \to (B,b)\to  (C,c) \cdots $$
The dangerous point $B$ is visited twice, and the two (residues of the) poles cancel each other in the sum $  F$.'

  \centering
 \includegraphics[scale=0.3]{ellipses.pdf}
 
 For some more information on the meromorphic technique, see  \cite{twohundred}. \cite{Cayley}, \cite{space}.
 
 \begin{thebibliography}{99}
 
\bibitem{twohundred}  Richard Schwartz andSergei Tabachnikov,  Centers of Mass of Poncelet Polygons, 200 Years After, Math Intelligencer (2016) 38:29-34.

\bibitem{Cayley} Phillip Griffiths and Joseph Harris, In Cayley's explicit solution to Poncelet's porism,
Colloquium on Topology and Algebra, Zurich, 1977.

\bibitem{space}   Phillip Griffiths and Joseph Harris, 
A Poncelet theorem in space,  Comment. Math. Helvetici 52 (1977) 145-160.



\end{thebibliography}







\end{document}