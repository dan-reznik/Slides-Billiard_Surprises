\documentclass[11pt]{article}
\usepackage{url}
\usepackage{graphicx}	
\usepackage{color}
\usepackage[utf8]{inputenc}

%\usepackage{cleveref}
%\crefname{section}{¤}{¤¤}
%\Crefname{section}{¤}{¤¤}
\begin{document}

\begin{center}
{\Large {\bf    The sum of cosines by space averaging}} % via Birkhoff's ergodic principle }}
\end{center}


Fix a  value of  the integral  of motion  \begin{equation} r = (AX,V) \,.  \label{integral}  \end{equation}
    The  $r$'s  corresponding to closed orbits  could  be found implementing Cayley's conditions.
We present  instead a computation that holds for all $r$, so most correspond to quasi-periodic orbits.     Giving a starting point  $P_o$,  consider the {\it asymptotic} average 
 \begin{equation}     K(c) =  \lim_{k \to \infty}\,\,     (1/k) \,  \sum_{i=1}^k   \cos(\theta_i) .
 \end{equation}
 where  $\theta_i$ are the internal angles between the consecutive segments of the trajectory, that in general does not close.
For those  $r$ yielding irrational invariant tori      it is   intuitive  that $ K(r) $  does not depend on the starting point on the  ellipse boundary.
 In fact the {\it averaging principle}  (see Arnold \cite{Arnold},
section 51, or Jovanovic  \cite{Jovanovic} for a short discussion)  gives
   \begin{equation}    K(r) =   \oint   \cos(\theta) \rho(s) \, ds  .
   \end{equation} 
where the integral is performed along the external boundary  and  $\rho$  is the density of an invariant measure  with total measure normalized to 1.
Roughly,  this measure  gives  the probability of a boundary point being hit as the trajectory is followed forever forwards and backwards in time.

This measure has an explicit expression.   % e recall a more general construction.  %  It  is more interesting  to do the averaging along the caustic (the internal ellipse).  %  Going back to the 1950's and 1970's
  Given a convex curve $C$,  the  string construction consists of enveloping the curve  tightly with a string
of length $ L(C) + p $.   This  yields a  surrounding curve $C_p$ that, {\it seen as a billiard boundary}, has $C$ as a caustic.  Poritsky  \cite{Poritsky} and Lazutkin \cite{Lazutkin}  showed that  the billiard map restricted to the segments
tangent to $C$ has an invariant measure 
\begin{equation}  \rho  =  dx = \kappa^{2/3} ds, 
\end{equation} 
where $\kappa$   is the curvature and  $s$ the arc length of  the caustic $C$.  In other words, $x \to  x + {\rm const.}$ linearizes the billiard map\footnote{See    Glutsyuk  \cite{Glutsyuk}  and \cite{Zhang} for recent work on the subject.}.  Surprisingly   (at least to us)   this measure is ``universal'' in the sense that it does not depend on $p$, just that constant depends on $p$. The case of  elliptic billiards is  developed   in detail in Izmestiev and  Tabachnikov  \cite{Izmestiev}.

\newpage
% The case of  elliptic billiards is  developed   in detail in Izmestiev and  Tabachnikov  \cite{Izmestiev}.  \\ \\


% Let an ellipse  $E$  be  the caustic.    $E$  has uniformizing  coordinate $ x,  dx =  \kappa^{2/3} ds,$   such that the billiard reflection in every confocal ellipse  $E$', containing E, %induces the map  $x \to  x +  \alpha $ (where the constant $\alpha$ depends on $E$'. \\
 % It is indeed amazing the  universality of  (\ref{k}).
 
\includegraphics[scale=0.3]{parameter_x.png}

$$ $$

Thus our  task is to compute along the caustic ellipse the space average 
            \begin{equation} \label{task}       K(r) =   \frac{ 1}{\oint \,  \kappa^{2/3}  ds} \,  \, \oint \,   \cos(\theta)\,  \kappa^{2/3} ds  .
            \end{equation}



The result of the computation of (\ref{task}), in blue,  compared with the   direct simulation with 500 bounces  (in green) is depicted in the figure below.  The match
is so perfect that one does not see the blue curve.  The behavior for small  $b_{cau}$   should be studied in more detail, as the average is still around 0.8 for  $b_{cau} =  0.01 $
and should rise  to the value 1  for  $b_{cau} =  0$. \\ \\ \\   \\


\includegraphics[scale=0.5]{comparison.jpg}

\noindent Follows the relevant formulae for the calculation.  We adopt  the ``russian'' notation for the ellipses,  using  the squares of the axis values, so
$$  E:\,\,\, \frac{x^2}{a} +  \frac{y^2}{b} = 1\,\,\, ,\,\,\,  E_{cau}:\,\,\,  \frac{x^2}{a_o} +  \frac{y^2}{b_o} = 1\,\,\, ,\,\,\,  a - a_o =  b - b_o = \lambda
$$
Some elementary geometry gives the relations between $r, \lambda $ and $b_{cau}$:
\begin{equation}  b_{cau} = \sqrt{b_o}\,,\,\, b_o = b - \lambda,\,\,\,  r^2 =  \frac{\lambda}{ab} \,\,. 
\end{equation}
The caustic is parametrized by  
$$  P_o:\,\,\,   x_o = \sqrt{a_o} \cos u\,,\, y_o = \sqrt{a_o} \sin u.
$$ 
The intersections with $E$ of the tangent line to the caustic  at $P_o$ are  the two points
\begin{equation}   P_{\pm} :    \,\,\,   x_{1,2} = x_o -  \frac{y_o}{b_o} \,\, t^*_{\pm} \,\,,\,\, y_{1,2} = y_o +  \frac{x_o}{a_o} \,\, t^*_{\pm}
\end{equation} 
where
\begin{equation} t^*_{\pm}  =  \frac{1}{A} \left(  B \pm  \sqrt{B^2 + AC}  \right)  
\end{equation}
with  $A, B, C $ given by     ($A, C > 0$)
\begin{equation}    A = \frac{x_o^2}{b a_o^2} +  \frac{y_o^2}{a b_o^2}\,\,,\,\,  B = x_o y_o \, (\frac{1}{a b_o} -  \frac{1}{b a_o}  )\,\,,\,\,
C = 1 -  \frac{x_o^2}{a} -  \frac{y_o^2}{b}  .
\end{equation} 
Using   the integral  (\ref{integral}) we get
\begin{equation} 
\cos \theta_{1,2} (u) =   2 r^2 \, \left( \frac{x_{1,2}^2}{a^2}  + \frac{y_{1,2}^2}{b^2}  \right)^{-1} \, -   1 \, ,
\end{equation} 
Finally the arc length and curvature along the caustic ellipse are given by
$$ ds =  \left( a_o \, \sin^2 u + b_o \,  \cos^2 u       \right)^{1/2} du
$$ 
$$   \kappa = \sqrt{a_o b_o} \, \left( a_o \, \sin^2 u + b_o \,  \cos^2 u       \right)^{-3/2}
$$ 
so that
\begin{equation}
\rho = dx =  \kappa^{2/3}\, ds =   (a_o b_o)^{1/3} \, \left( a_o \, \sin^2 u + b_o \,  \cos^2 u       \right)^{-1/2}
\end{equation}

\begin{thebibliography}{99}

\bibitem{Arnold}  V. Arnold, Mathematical Methods of Classical Mechanics, Springer Graduate Texts in Mathematics 60,  1978.

\bibitem{Jovanovic}  %Bo\v{z}idar  
B. Jovanovi\'c,  What are completely integrable Hamilton systems,  The Teachingg of Mathematics, v. 13:1,  1-14, 2011 (\url{http://elib.mi.sanu.ac.rs/files/journals/tm/26/tm1411.pdf}).

\bibitem{Poritsky} H.  Poritsky,  The billiard ball problem on a table with a convex boundary -  an illustrative dynamical problem,  Ann. of Math. (2) 51 (1950), 446-470.


\bibitem{Lazutkin} V. F. Lazutkin,The existence of caustics for a billiard problem in a convex domain, 	Izv. Akad. Nauk SSSR Ser. Mat., 1973,	v. 3:1,	 186-216.	 

\bibitem{Glutsyuk}	A. Glutsyuk,  On curves with Poritsky property,  \url{https://arxiv.org/abs/1901.01881}



\bibitem{Zhang}  J. Zhang, Suspension of the Billiard maps in the Lazutkin's coordinate, \url{ https://arxiv.org/abs/1610.00317}

\bibitem{Izmestiev} I.Izmestiev,  S.Tabachnikov,  Ivorys theorem revisited,  
Journal of Integrable Systems,  2:1, January 2017,  \url{https://doi.org/10.1093/integr/xyx006}
 





\end{thebibliography}


\end{document}





\begin{center}
{\Large {\bf    The function   $\cos \theta $  MUST REDO... }}
\end{center}



\noindent From basic analytic geometry:   the angle between lines of slopes $m_1, m_2$  is
$$ \tan \theta  =  \pm \frac{ m_1 - m_2  } {1 + m_1 m_2 } \,\,\,\,  \Rightarrow \,\,\,\, 
|\cos \theta| =      \frac{ | 1 + m_1 m_2 |  } {\sqrt{(1 + m_1^2)(1 + m_2^2)} }
 $$
It can be rewritten as
\begin{equation}  \label{cos}
| \cos \theta | =  \    \frac{ | 1 + m_1 m_2  | } {\sqrt{(1 - m_1m_2)^2 + (m_1 + m_2)^2} }
\end{equation} 

From  the sources below, given a point  $(x_o,y_o)$ outside an ellipse    $$x^2/a^2  + y^2/b^2  = 1 $$ the tangent lines  $ y = m x + d$  are determined by the system
$$ y_o = m x_o + q \,\,\,  ,\,\,\,  a^2 m^2 + b^2 =  q^2  \,\,. 
$$
 It reduces to a quadratic equation.   There are two pairs of solutions
$m_1, q_1$ and  $m_2, q_2 $. 
Once they are computed, the two points of contact are given by   
\begin{equation} ( -a^2 m_i/q_i \,,\, b_i^2/q_i  ) \, , \,\,\,\, i = 1,2.
\end{equation}

The equation for $m_1 , m_2 $ is
$$  (a^2 - x_o^2) m^2 + 2 x_o y_o \, m + b^2 - y_o = 0
$$
so that
\begin{equation}  \label{m}   m_1 + m_2 = - \frac{ 2 x_o y_o  }{ a^2 - x_o^2}\,\,,\,\, m_1 m_2  = \frac{b^2 - y_o^2  }{ a^2 - x_o^2}
\end{equation}
 \smallskip 

\noindent Source:\\

\noindent \url{http://www.nabla.hr/CS-ContentsA.htm}\\
\url{http://www.nabla.hr/CS-EllipseAndLine1.htm}\\
\url{http://www.nabla.hr/CS-EllipseAndLine2.htm}\\

The desired expression  for  $\cos \theta (t)  $ results by inserting  (\ref{m}) into (\ref{cos})  with 
%\centerline {  
 $ (a \cos t \,,  b \sin t) $    inputs to   $ (x_o, y_o)$
%  }
and 
%\centerline   { 
$  a^2 - \lambda\,\,,\,\,  b^2 - \lambda  $  replacing    $  a,  b  $  in the above formulas.
% }
\bigskip

\noindent  As we shown before,  $ \lambda = ab c $  where $c = (AX,  \hat{v} ) = c $   is the second integral of motion,  and where we take  the particle with unit velocity. \\ \\ 

Now to set up  the integral  one needs to work  backwards, parametrizing the point of contact. 
