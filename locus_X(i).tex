\documentclass[11pt]{amsart}

\usepackage{amsbsy, amsmath,amsfonts,amssymb,amsthm,amscd,amssymb,latexsym}
\usepackage[latin1]{inputenc}
 \usepackage[brazil]{babel}
\usepackage{graphicx}
\usepackage[dvips]{epsfig}
\usepackage{graphics}
\usepackage{psfrag} 
\usepackage[usenames,dvipsnames]{xcolor}
 
 
 
 

    \linespread{1.3}
\usepackage{cancel}

\usepackage{enumerate}
\usepackage{indentfirst}
\usepackage{euscript} 
 

%  \usepackage{refcheck}


\tolerance= 2000
 



 \theoremstyle{plain}
 \newtheorem{theorem}{Teorema}
%\newtheorem{proof}{Demonstra\cao}
\newtheorem{proposition}{Proposi\cao}
\newtheorem{lemma}{Lema}
\newtheorem{corollary}{Corol\'ario}
 \newtheorem{remark}{Observa\cao}
\newtheorem{example}{Exemplo}

 %\newtheorem{proof}{Demonstra\cao}
%\newtheorem{lemma}{Lema}
 \theoremstyle{definition}
 \newtheorem{definition}{Defini\c c\~ao}
% \newtheorem{proof}
%  \newtheorem{proof}{Demonstra\c c\~ao}
\newcommand{\po}{\partial}
\newcommand{\cqd}{\hfill\fbox{ }}
\newcommand{\cal}{\mathcal}

 \newtheorem{problem}{Problema}
%\newtheorem{abstract}{Resumo}

\def \nota{{\bigskip\bf NOTA:}}
\def \eg{e.g.}
\def \prx{\frac {\partial}{\partial x}}
\def \prs{\frac {\partial}{\partial s}}
\def \prt{\frac {\partial}{\partial t}}
\def \ie{{\it {i.e.}}}

\def\sen{\,\text{sen}\,}



\def \coes{\c c\~oes }
\def \ii{\'\i }
\def \cao{\c c\~ao }
\def \sao{s\~ao }
 \def \oes{\~oes }
\def \ao{\~ao }
 \def \ue{\"u\^e}

 \title[X(i) locus ]{ $X(i)$   loci of billiards}



 \author{  Ronaldo A. Garcia  }

% \thanks{   Veja www.mat.ufg.br} %




 \date{\today }

 
 

 \begin{document}


  \begin{abstract}  $X(6)$ is given by a quartic equation
  	\vskip .2cm
  	
   
  	
  	 %	\noindent {Keywords: ellipse, billiard, affine curvature, circumcenter }
\end{abstract}
\maketitle


 


% \section{Introdu\c c\~ao }
 
 
 
 
  \section{Locus X(1)}
  
 The ellipse \[ \frac{x^2}{a_{1}^2}+\frac{y^2}{b_{1}^2}=1, \]
 where
 \[\aligned 
 a_1=&   \frac {-{b}^{2}+\sqrt {{a}^{4}-{a}^{2}{b}^{2}+{b}^{4}}}{a}
 %-y5;
 \\
 b_1=& -\frac {-{a}^{2}+\sqrt {{a}^{4}-{a}^{2}{b}^{2}+{b}^{4}}}{b}
 %
 \endaligned\] 
 is the geometric locus of $X(1)$.
 
 The confocal ellipse is given
 
 \[\aligned  
a_c= &- \frac{a \left( {b}^{2}-\sqrt {{a}^{4}-{a}^{2}{b}^{2}+{b}^{4}}
 		\right) }{{a}^{2}-{b}^{2}}\\
 b_c= &\frac{b \left( {a}^{2}-\sqrt {{a}^{4}-{a}^{2}{b}^{2}+{b}^{4}}
 		\right) }{{a}^{2}-{b}^{2}}
 \endaligned \]
 
 
 The excentric ellipse 
 
  \[\aligned  
 a_e= & {\frac {{b}^{2}+\sqrt{a^4-a^2b^2+b^4}}{a}}  \\
 b_e= & {\frac {{a}^{2}+\sqrt{a^4-a^2b^2+b^4}}{b}},\;\; \delta=\sqrt{a^4-a^2b^2+b^4}
 \endaligned \]
 
 \section{Locus X(2)-Circuncenter}
 
 The ellipse \[ \frac{x^2}{a_{2}^2}+\frac{y^2}{b_{2}^2}=1, \]
 where
 \[\aligned 
 a_2=&    \frac 12\,{\frac {{a}^{2}-\sqrt {{a}^{4}-{a}^{2}{b}^{2}+{b}^{4}}}{a}}
 %-y5;
 \\
 b_2=&  -\frac 12 \,{\frac {{b}^{2}-\sqrt {{a}^{4}-{a}^{2}{b}^{2}+{b}^{4}}}{b}}
 %
 \endaligned\] 
 is the geometric locus of $X(2)$.
 

 
  \section{Locus X(3)-Barycenter}
 
 The ellipse \[ \frac{x^2}{a_{3}^2}+\frac{y^2}{b_{3}^2}=1, \]
 where
 \[\aligned 
 a_3^2=&      {\frac {{a}^{2} \left( 5\,{a}^{4}+5\,{b}^{4}- \left( 4\,{a}^{2}+4
 		\,{b}^{2} \right) \sqrt {{a}^{4}-{a}^{2}{b}^{2}+{b}^{4}}-2\,{a}^{2}{b}
 		^{2} \right) }{ 9\left( {a}^{2}-{b}^{2} \right) ^{2}}}\\
 	a_3=& \,{\frac {a \left( -{a}^{2}-{b}^{2}+2\,\sqrt {{a}^{4}-{a}^{2}{b}^{2
 				}+{b}^{4}} \right) }{3 \left( a^2-b^2 \right)    }}
 %	
 %
 \\
 b_3^2=&   \,{\frac {{b}^{2} \left( 5\,{a}^{4}+5\,{b}^{4}- \left( 4\,{a}^{2}+4
 		\,{b}^{2} \right) \sqrt {{a}^{4}-{a}^{2}{b}^{2}+{b}^{4}}-2\,{a}^{2}{b}
 		^{2} \right) }{ 9\left( {a}^{2}-{b}^{2} \right) ^{2}}}\\
 	b_3=&  \,{\frac { \left( -{a}^{2}-{b}^{2}+2\,\sqrt {{a}^{4}-{a}^{2}{b}^{2}
 				+{b}^{4}} \right) b}{3 \left( a^2-b^2 \right)    }}
 	%
  %
 \endaligned\] 
 is the geometric locus of $X(3)$.
 
  \section{Locus X(4)-Ortocenter}
 
 The ellipse \[ \frac{x^2}{a_{4}^2}+\frac{y^2}{b_{4}^2}=1, \]
 where
 \[\aligned 
 a_4^2=&    {\frac {-4\,{a}^{2}{b}^{2} \left( {a}^{2}+{b}^{2} \right) \sqrt {{a}^{
 				4}-{a}^{2}{b}^{2}+{b}^{4}}+ \left( {a}^{4}+{b}^{4} \right) ^{2}+{a}^{2
 		}{b}^{2} \left( {a}^{2}+{b}^{2} \right) ^{2}}{{a}^{2} \left( {a}^{2}-{
 			b}^{2} \right) ^{2}}}
 %
 %-y5;
 \\
 b_4^2=&  {\frac {-4\,{a}^{2}{b}^{2} \left( {a}^{2}+{b}^{2} \right) \sqrt {{a}^{
 				4}-{a}^{2}{b}^{2}+{b}^{4}}+ \left( {a}^{4}+{b}^{4} \right) ^{2}+{a}^{2
 		}{b}^{2} \left( {a}^{2}+{b}^{2} \right) ^{2}}{{b}^{2} \left( {a}^{2}-{
 			b}^{2} \right) ^{2}}}\\
 		a_4=&{\frac { \left( {a}^{2}+{b}^{2} \right) \sqrt {{a}^{4}-{b}^{2}{a}^{2}+
 					{b}^{4}}-2\,{b}^{2}{a}^{2}}{\left({a}^{2}-{b}^{2}\right)a}}\\
 b_4=&{\frac { \left( {a}^{2}+{b}^{2} \right) \sqrt {{a}^{4}-{b}^{2}{a}^{2}+
 {b}^{4}}-2\,{b}^{2}{a}^{2}}{\left({a}^{2}-{b}^{2}\right)b}}
 %
 \endaligned 
 \] 
 is the geometric locus of $X(4)$.
 
 $  b_4=b $  if and only if  $b=\frac 17 \,\sqrt {7+14\,\sqrt {2}}a=(0.7395391544...)a$
 
  $  b_4=a $  if and only if  $b=(0.6622640780...)a$
  where  
   ${a}^{6}+{a}^{4}{b}^{2}-4\,{a}^{3}{b}^{3}-{a}^{2}{b}^{4}-{b}^{6}=0.$
 
 
% \section{Locus X(6)- Semymedian}
% 
%   \section{Locus X(5)- NINE-POINT CENTER}
% 
% The ellipse \[ \frac{x^2}{a_{5}^2}+\frac{y^2}{b_{5}^2}=1, \]
% where
% \[\aligned 
% a_5=&     \,{\frac {-{a}^{2} \left( {a}^{2}+3\,{b}^{2} \right) +\sqrt {{a}^{4
% 			}-{a}^{2}{b}^{2}+{b}^{4}} \left( 3\,{a}^{2}+{b}^{2} \right) }{ 4a\left( 
% 		{a}^{2}-{b}^{2} \right)  }}\\
% a_5	=& \,{\frac {-{a}^{2} \left( {a}^{2}+3\,{b}^{2} \right) +\sqrt {{a}^{4
% 				}-{b}^{2}{a}^{2}+{b}^{4}} \left( 3\,{a}^{2}+{b}^{2} \right) }{4a
% 			\left( {a}^{2}-{b}^{2} \right) }}
%% 	
% 	%
% %
% %-y5;
% \\
% b_5=&   \,{\frac {-{b}^{2} \left( 3\,{a}^{2}+{b}^{2} \right) +\sqrt {{a}^{4
% 			}-{a}^{2}{b}^{2}+{b}^{4}} \left( {a}^{2}+3\,{b}^{2} \right) }{4b
% 		\left( {a}^{2}-{b}^{2} \right) }}
% %
% %
% \endaligned\] 
% is the geometric locus of $X(5)$.
 
  \section{Locus X(6)- Semymedian}
  
  
 The locus $X(6)$ is the quartic defined by:
 
 
 \[\aligned Q_4(x,y)=&
 {b}^{4} \left( 5\,{a}^{4}-6\,{a}^{2}{b}^{2}+5\,{b}^{4}-4\,{a}^{2}
 \delta+4\,{b}^{2}\delta \right) {x}^{4}-{a}^{2}{b}^{4} \left( 5\,{a}^{
 	4}-5\,{a}^{2}{b}^{2}+2\,{b}^{4}-4\,{a}^{2}\delta+2\,{b}^{2}\delta
 \right) {x}^{2}\\
 +&{a}^{4} \left( 5\,{a}^{4}-6\,{a}^{2}{b}^{2}+5\,{b}^{4
 }+4\,{a}^{2}\delta-4\,{b}^{2}\delta \right) {y}^{4}-{a}^{4}{b}^{2}
 \left( 2\,{a}^{4}-5\,{a}^{2}{b}^{2}+5\,{b}^{4}+2\,{a}^{2}\delta-4\,{b
 }^{2}\delta \right) {y}^{2}\\
+&2\,{b}^{2}{a}^{2} \left( 3\,{a}^{4}-2\,{a}
 ^{2}{b}^{2}+3\,{b}^{4} \right) {x}^{2}{y}^{2}\\
 \delta=&\sqrt {{a}^{4}-{a}^{2}{b}^{2}+{b}^{4}}\endaligned
 \]
 
 
 The ellipse \[ \frac{x^2}{a_6^2}+\frac{y^2}{b_6^2}=1, \]
 where
\[\aligned  a_6=& {\frac { \left( -{a}^{2}{b}^{2}-{b}^{4}+3\,\sqrt {{a}^{4}-{a}^{2}{b}^{
  				2}+{b}^{4}}{a}^{2}-\sqrt {{a}^{4}-{a}^{2}{b}^{2}+{b}^{4}}{b}^{2}
  		\right) a}{3\,{a}^{4}-2\,{a}^{2}{b}^{2}+3\,{b}^{4}}}\\
  	b_6=&{\frac { \left( {a}^{4}+{a}^{2}{b}^{2}+\sqrt {{a}^{4}-{a}^{2}{b}^{2}+{
  					b}^{4}}{a}^{2}-3\,\sqrt {{a}^{4}-{a}^{2}{b}^{2}+{b}^{4}}{b}^{2}
  			\right) b}{3\,{a}^{4}-2\,{a}^{2}{b}^{2}+3\,{b}^{4}}}
  		\endaligned
   \]
   is tangent to the quartic $Q_4(x,y)=0.$
   
  \section{Locus X(5)}
   The ellipse \[ \frac{x^2}{a_{5}^2}+\frac{y^2}{b_{5}^2}=1, \]
  where
  \[\aligned 
  a_5=&   \frac {{a}^{2} \left( {a}^{2}+3\,{b}^{2} \right) -\sqrt {{a}^{4}
  		-{a}^{2}{b}^{2}+{b}^{4}} \left( 3\,{a}^{2}+{b}^{2} \right) }{4a \left( 
  	{a}^{2}-{b}^{2} \right) }
%-y5;
 \\
   	b_5=&{\frac {-{b}^{2} \left( 3\,{a}^{2}+{b}^{2} \right) +\sqrt {{a}^{4
   				}-{a}^{2}{b}^{2}+{b}^{4}} \left( {a}^{2}+3\,{b}^{2} \right) }{4b
   			\left( {a}^{2}-{b}^{2} \right) }}
   	%
   	\endaligned\] 
   is the geometric locus of $X(5)$.
     
    
  The conversion formula from the trilinear coordinates x, y, z to the vector of Cartesian coordinates P of the point is given by
  
   
  \[ \aligned   
  P=&{\frac {ax}{ax+by+cz}}{  A}+{\frac {by}{ax+by+cz}}{  {B}}+{\frac {cz}{ax+by+cz}}{ C} 
\endaligned\]
   
   where the side lengths are $|C - B| = a$, $|A - C| = b$ and $|B - A| = c. $
    \section{Locus X(7) -- Gergonne Point}
    
    
   
   The ellipse \[ \frac{x^2}{a_{7}^2}+\frac{y^2}{b_{7}^2}=1, \]
   where
   \[\aligned 
   a_7=&    -{\frac { \left( {a}^{2}+{b}^{2}-2\,\sqrt {{a}^{4}-{a}^{2}{b}^{2}+{b}^
   			{4}} \right) a}{{a}^{2}-{b}^{2}}}\\
   %
   %-y5;
   \\
   b_7=&  -{\frac { \left( {a}^{2}+{b}^{2}-2\,\sqrt {{a}^{4}-{a}^{2}{b}^{2}+{b}^
   			{4}} \right) b}{{a}^{2}-{b}^{2}}}
   %
   %
   \endaligned\] 
   is the geometric locus of $X(7)$. Is is similar to the billiard (ellipse).
   
   
         \section{Locus X(8) -- Nagel Point}
   
   The ellipse \[ \frac{x^2}{a_{8}^2}+\frac{y^2}{b_{8}^2}=1, \]
   where
   \[\aligned 
   a_{8}=&   -{\frac {-{a}^{4}+{a}^{2}{b}^{2}-2\,{b}^{4}+2\,b^2\sqrt {{a}^{4}-{a}^{2}{
   				b}^{2}+{b}^{4}} }{a \left( a^2-b^2\right)   }}
   %
   \\
   b_{8}=&  {\frac {-2\,{a}^{4}+{a}^{2}{b}^{2}-{b}^{4}+2\,{a}^{2} \sqrt {{a}^{4}-{a}^{2}{b
   			}^{2}+{b}^{4}}}{b \left( {a}^{2}-{b}^{2} \right) }}
   %
   \endaligned\] 
   is the geometric locus of $X(8)$.
   
   
      \section{Locus X(10)}
      
       The ellipse \[ \frac{x^2}{a_{10}^2}+\frac{y^2}{b_{10}^2}=1, \]
      where
      \[\aligned 
      a_{10}=&    {\frac { \left( {a}^{2}+{b}^{2} \right) \sqrt {{a}^{4}-{a}^{2}{b}
      			^{2}+{b}^{4}}-{a}^{4}-{b}^{4}}{ 2\left( {a}^{2}-{b}^{2} \right) a}}
\\
b_{10}=& \,{\frac { \left( {a}^{2}+{b}^{2} \right) \sqrt {{a}^{4}-{a}^{2}{b}
			^{2}+{b}^{4}}-{a}^{4}-{b}^{4}}{ 2\left( {a}^{2}-{b}^{2} \right) b}}
%
\endaligned\] 
is the geometric locus of $X(10)$.

 \section{  X(11)  {Feuerbach Point} }
 
 It is the caustic

\section{Locus X(12)  {X(1),X(5)}-HARMONIC CONJUGATE OF X(11) }

The ellipse \[ \frac{x^2}{a_{12}^2}+\frac{y^2}{b_{12}^2}=1, \]
where
\[\aligned 
a_{12}=&     -{\frac {-{b}^{2} \left( 15\,{a}^{6}+12\,{b}^{2}{a}^{4}+3\,{a}^{2}{b}^
		{4}+2\,{b}^{6} \right) + \left( 7\,{a}^{6}+12\,{b}^{2}{a}^{4}+11\,{a}^
		{2}{b}^{4}+2\,{b}^{6} \right) \sqrt {{a}^{4}-{a}^{2}{b}^{2}+{b}^{4}}}{
		a \left( 7\,{a}^{6}+11\,{b}^{2}{a}^{4}-11\,{a}^{2}{b}^{4}-7\,{b}^{6}
		\right) }}
%
\\
b_{12}=&  {\frac {-2\,{a}^{8}-3\,{b}^{2}{a}^{6}-12\,{a}^{4}{b}^{4}-15\,{b}^{6}{a
		}^{2}+ \left( 2\,{a}^{6}+11\,{b}^{2}{a}^{4}+12\,{a}^{2}{b}^{4}+7\,{b}^
		{6} \right) \sqrt {{a}^{4}-{a}^{2}{b}^{2}+{b}^{4}}}{b \left( 7\,{a}^{6
		}+11\,{b}^{2}{a}^{4}-11\,{a}^{2}{b}^{4}-7\,{b}^{6} \right) }}
%
%
\endaligned\] 
is the geometric locus of $X(12)$.

%     \section{Locus X(35)}
%
%The ellipse \[ \frac{x^2}{a_{35}^2}+\frac{y^2}{b_{35}^2}=1, \]
%where
%\[\aligned 
%a_{35}=&     
%\\
%b_{35}=&  
%%%
%%\endaligned\] 
%%is the geometric locus of $X(35)$.
%
%     \section{Locus X(35)}
%
%The ellipse \[ \frac{x^2}{a_{35}^2}+\frac{y^2}{b_{35}^2}=1, \]
%where
%\[\aligned 
%a_{35}=&     
%\\
%b_{35}=&  
%%
%\endaligned\] 
%is the geometric locus of $X(35)$.

     \section{Locus X(20)- DE LONGCHAMPS POINT }

The ellipse \[ \frac{x^2}{a_{20}^2}+\frac{y^2}{b_{20}^2}=1, \]
where
\[\aligned 
a_{20}=&     {\frac {-{a}^{2} \left( {a}^{2}-3\,{b}^{2} \right) -2b^2\,\sqrt {{a}^{4}-
			{a}^{2}{b}^{2}+{b}^{4}} }{a(  {a}^{2}- {b}^{2})}}
%
\\
b_{20}=&  -{\frac {{b}^{2} \left( 3\,{a}^{2}-{b}^{2} \right) -2a^2\,\sqrt {{a}^{4}-
			{a}^{2}{b}^{2}+{b}^{4}} }{b \left( {a}^{2}-{b}^{2} \right) }}
%
%
\endaligned\] 
is the geometric locus of $X(20)$.

     \section{Locus X(21)}

The ellipse \[ \frac{x^2}{a_{21}^2}+\frac{y^2}{b_{21}^2}=1, \]
where
\[\aligned 
a_{21}=&     {\frac {-{a}^{4}-{a}^{2}{b}^{2}-2\,{b}^{4}+2\, \left( {a}^{2}+{b}^{2}
		\right) \sqrt {{a}^{4}-{a}^{2}{b}^{2}+{b}^{4}}}{a \left( 3\,{a}^{2}+5
		\,{b}^{2} \right) }}
%
\\
b_{21}=&  {\frac {2\,{a}^{4}+{a}^{2}{b}^{2}+{b}^{4}-2\, \left( {a}^{2}+{b}^{2}
		\right) \sqrt {{a}^{4}-{a}^{2}{b}^{2}+{b}^{4}}}{b \left( 5\,{a}^{2}+3
		\,{b}^{2} \right) }}
%
%
\endaligned\] 
is the geometric locus of $X(21)$.



     \section{Locus X(35)}

The ellipse \[ \frac{x^2}{a_{35}^2}+\frac{y^2}{b_{35}^2}=1, \]
where
\[\aligned 
a_{35}=&     {\frac { \left( 7\,{a}^{2}+{b}^{2} \right)  \left( {a}^{2}+{b}^{2}
		\right) \sqrt {{a}^{4}-{a}^{2}{b}^{2}+{b}^{4}}-{b}^{2} \left( 11\,{a}
		^{4}+4\,{a}^{2}{b}^{2}+{b}^{4} \right) }{a \left( 7\,{a}^{4}+18\,{a}^{
			2}{b}^{2}+7\,{b}^{4} \right) }}
%
\\
b_{35}=&  {\frac { \left( {a}^{2}+7\,{b}^{2} \right)  \left( {a}^{2}+{b}^{2}
		\right) \sqrt {{a}^{4}-{a}^{2}{b}^{2}+{b}^{4}}-{a}^{2} \left( {a}^{4}
		+4\,{a}^{2}{b}^{2}+11\,{b}^{4} \right) }{b \left( 7\,{a}^{4}+18\,{a}^{
			2}{b}^{2}+7\,{b}^{4} \right) }}
%
%
\endaligned\] 
is the geometric locus of $X(35)$.

     \section{Locus X(36) -- Inverse-in-Circumcircle of Incenter }

The ellipse \[ \frac{x^2}{a_{36}^2}+\frac{y^2}{b_{36}^2}=1, \]
where
\[\aligned 
a_{36}=&    \,{\frac { \left( 3\,{a}^{2}-{b}^{2} \right) \sqrt {{a}^{4}-{a}^{2}
			{b}^{2}+{b}^{4}}+{b}^{2} \left( {a}^{2}+{b}^{2} \right) }{3a \left( a^2-b^2
		\right)    }}
%
\\
b_{36}=&   {\frac { \left( {a}^{2}-3\,{b}^{2} \right) \sqrt {{a}^{4}-{a}^{2}{b}^{
				2}+{b}^{4}}+{a}^{2} \left( {a}^{2}+{b}^{2} \right) }{3b(  \,{a}^{2} - \,{b
		}^{2})}}
%
\endaligned\] 
is the geometric locus of $X(36)$.





      \section{Locus X(40)}

The ellipse \[ \frac{x^2}{a_{40}^2}+\frac{y^2}{b_{40}^2}=1, \]
where
\[\aligned 
a_{40}=&    \frac { a^2-b^2}{a  }
\\
b_{40}=&  \frac { a^2-b^2}{b  }
%
\endaligned\] 
is the geometric locus of $X(40)$.
      
      
       \section{Locus X(46)}
      
      The ellipse \[ \frac{x^2}{a_{46}^2}+\frac{y^2}{b_{46}^2}=1, \]
      where
      \[\aligned 
      a_{46}=&          
     {\frac { \left( {a}^{2}-{b}^{2} \right)  \left(  \left( 5\,{a}^{2}+{b}
     		^{2} \right) \sqrt {{a}^{4}-{a}^{2}{b}^{2}+{b}^{4}}+{b}^{2} \left( 3\,
     		{a}^{2}-{b}^{2} \right)  \right) }{a \left( 5\,{a}^{4}-6\,{a}^{2}{b}^{
     			2}+5\,{b}^{4} \right) }}
     %
      \\
      b_{46}=&  -{\frac { \left( {a}^{2} \left( {a}^{2}-3\,{b}^{2} \right) -\sqrt {{a}
      			^{4}-{a}^{2}{b}^{2}+{b}^{4}} \left( {a}^{2}+5\,{b}^{2} \right) 
      		\right)  \left( {a}^{2}-{b}^{2} \right) }{b \left( 5\,{a}^{4}-6\,{a}^
      		{2}{b}^{2}+5\,{b}^{4} \right) }}
      %
      %
      \endaligned\] 
      is the geometric locus of $X(40)$.

      
      \section{Locus X(55) --  }
      
      
      The ellipse 
      
      \[ \frac{x^2}{a_{55}^2}+\frac{y^2}{b_{55}^2}=1, \]
      where
      \[   \aligned
      a_{55}=&  {\frac { \left( -{b}^{2}+\sqrt {{a}^{4}-{b}^{2}{a}^{2}+{b}^{4}}
      		\right) a}{{a}^{2}+{b}^{2}}}
       \\
       b_{55} =&  {\frac { \left( {a}^{2}-\sqrt {{a}^{4}-{b}^{2}{a}^{2}+{b}^{4}}
       	\right) b}{{a}^{2}+{b}^{2}}}
      %
      %
      \endaligned
      \]
      
      is the geometric locus of $X(55)$.  It is similar to the confocal ellipse. 
      
      
      \section{Locus X(56) --  }
      
      
      The ellipse 
      
      \[ \frac{x^2}{a_{56}^2}+\frac{y^2}{b_{56}^2}=1, \]
      where
      \[   \aligned
      a_{ 56}=&  {\frac {-{b}^{2} \left( {a}^{4}-{a}^{2}{b}^{2}+2\,{b}^{4} \right) +
      		\left( 5\,{a}^{4}-5\,{a}^{2}{b}^{2}+2\,{b}^{4} \right) \sqrt {{a}^{4}
      			-{a}^{2}{b}^{2}+{b}^{4}}}{a \left( 5\,{a}^{4}-6\,{a}^{2}{b}^{2}+5\,{b}
      		^{4} \right) }}
      %
      %
      \\
      b_{56}=&  -{\frac {-{a}^{2} \left( 2\,{a}^{4}-{a}^{2}{b}^{2}+{b}^{4} \right) +
      		\left( 2\,{a}^{4}-5\,{a}^{2}{b}^{2}+5\,{b}^{4} \right) \sqrt {{a}^{4}
      			-{a}^{2}{b}^{2}+{b}^{4}}}{b \left( 5\,{a}^{4}-6\,{a}^{2}{b}^{2}+5\,{b}
      		^{4} \right) }}
      %
      %
      %
      \endaligned
      \]
      
      is the geometric locus of $X(56)$.
      
      \section{Locus X(57) -- 
      	X(57)  ISOGONAL CONJUGATE OF X(9)}
   \[   
     a_{57}=\frac{a (a^2- b^2)}{ \sqrt{ a^4-a^2 b^2+b^4}},\;\;\;
       b_{57}=\frac{b (a^2- b^2)}{ \sqrt{ a^4-a^2 b^2+b^4}}
     \]
     is the geometric locus of $X(57)$.
          
     \section{Locus X(63) }
     \[   \aligned
     a_{ 63}=& {\frac {a \left( {a}^{2}-{b}^{2} \right) }{{a}^{2}+{b}^{2}}}\\
     b_{63}=& {\frac {b \left( {a}^{2}-{b}^{2} \right) }{{a}^{2}+{b}^{2}}}
     \endaligned
     \]
     
       is the geometric locus of $X(63)$.
     
      \section{Locus X(65) }
    \[   \aligned
     a_{ 65}=&-{\frac {-{a}^{4}{b}^{2}-{a}^{2}{b}^{4}-2\,{b}^{6}- \left( {a}^{4}-3\,
     		{a}^{2}{b}^{2}-2\,{b}^{4} \right) \sqrt {{a}^{4}-{a}^{2}{b}^{2}+{b}^{4
     	}}}{a \left( a^2-b^2 \right)^{2}  }}\\
     b_{65}=&{\frac {-2\,{a}^{6}-{a}^{4}{b}^{2}-{a}^{2}{b}^{4}+ \left( 2\,{a}^{4}+3
     		\,{b}^{2}{a}^{2}-{b}^{4} \right) \sqrt {{a}^{4}-{b}^{2}{a}^{2}+{b}^{4}
     	}}{b \left( {a}^{2}-{b}^{2} \right) ^{2}}}
     \endaligned
     \]
          is the geometric locus of $X(65)$.
          
            \section{Locus X(72) --  }
          
          
          The ellipse 
          
          \[ \frac{x^2}{a_{72}^2}+\frac{y^2}{b_{72}^2}=1, \]
          where
          \[   \aligned
          a_{ 72}=&  {\frac {{a}^{6}+2\,{a}^{2}{b}^{4}+{b}^{6}-\sqrt {{a}^{4}-{a}^{2}{b}^{2
          			}+{b}^{4}}{b}^{2} \left( 3\,{a}^{2}+{b}^{2} \right) }{a \left( a^2-b^2
          		\right) ^{2}  }}
          %
          %
          \\
          b_{72}=&  {\frac {-{a}^{6}-2\,{a}^{4}{b}^{2}-{b}^{6}+{a}^{2} \left( {a}^{2}+3\,{
          			b}^{2} \right) \sqrt {{a}^{4}-{a}^{2}{b}^{2}+{b}^{4}}}{b \left( a^2-b^2
          		\right) ^{2}  }}
          %
          %
          \endaligned
          \]
          
          is the geometric locus of $X(72)$.
          
            \section{Locus X(78)  --  ISOGONAL CONJUGATE OF X(34)}
          
          
          The ellipse 
          
          \[ \frac{x^2}{a_{78}^2}+\frac{y^2}{b_{78}^2}=1, \]
          where
          \[   \aligned
          a_{ 78}=&  {\frac {5\,{a}^{6}-4\,{a}^{4}{b}^{2}+{a}^{2}{b}^{4}+2\,{b}^{6}-2\,
          		\sqrt {{a}^{4}-{b}^{2}{a}^{2}+{b}^{4}}{b}^{2} \left( {a}^{2}+{b}^{2}
          		\right) }{a \left( 5\,{a}^{4}-6\,{b}^{2}{a}^{2}+5\,{b}^{4} \right) }}
          %
          %
          \\
          b_{78}=&  {\frac {-2\,{a}^{6}-{a}^{4}{b}^{2}+4\,{a}^{2}{b}^{4}-5\,{b}^{6}+2\,
          		\sqrt {{a}^{4}-{b}^{2}{a}^{2}+{b}^{4}}{a}^{2} \left( {a}^{2}+{b}^{2}
          		\right) }{b \left( 5\,{a}^{4}-6\,{b}^{2}{a}^{2}+5\,{b}^{4} \right) }}
          %
          %
          %
          \endaligned
          \]
          
          is the geometric locus of $X(78)$.
          
       
     \section{Locus X(79) -- ISOGONAL CONJUGATE OF X(35) }
     
     
     The ellipse 
     
     \[ \frac{x^2}{a_{79}^2}+\frac{y^2}{b_{79}^2}=1, \]
     where
     \[   \aligned
     a_{ 79}=&  {\frac {-{b}^{2} \left( 11\,{a}^{4}+4\,{b}^{2}{a}^{2}+{b}^{4} \right) 
     		+ \left( 3\,{a}^{4}+12\,{b}^{2}{a}^{2}+{b}^{4} \right) \sqrt {{a}^{4}-
     			{b}^{2}{a}^{2}+{b}^{4}}}{a \left( a^2-b^2 \right)  
     		\left( 3\,{a}^{2}+5\,{b}^{2} \right) }}
     %
     %
     \\
     b_{79}=&  {\frac {-{a}^{2} \left( {a}^{4}+4\,{b}^{2}{a}^{2}+11\,{b}^{4} \right) 
     		+ \left( {a}^{4}+12\,{b}^{2}{a}^{2}+3\,{b}^{4} \right) \sqrt {{a}^{4}-
     			{b}^{2}{a}^{2}+{b}^{4}}}{b \left( a^2-b^2  \right)  
     		\left( 5\,{a}^{2}+3\,{b}^{2} \right) }}
     %
     %
     %
     \endaligned
     \]
     
     is the geometric locus of $X(79)$.
%       
%     \section{Locus X(80) -- REFLECTION OF INCENTER IN FEUERBACH POINT}
%     
%     
%     The ellipse 
%     
%     \[ \frac{x^2}{a_{80}^2}+\frac{y^2}{b_{80}^2}=1, \]
%     where
%     \[   \aligned
%     a_{ 80}=&   {\frac { \left( -{b}^{2}+\sqrt {{a}^{4}-{b}^{2}{a}^{2}+{b}^{4}}
%     		\right)  \left( {a}^{2}+{b}^{2} \right) }{\left({a}^{2}-{b}^{2}\right) a}}
%     %
%     %
%     \\
%     b_{80}=&  {\frac { \left( {a}^{2}-\sqrt {{a}^{4}-{b}^{2}{a}^{2}+{b}^{4}}
%     		\right)  \left( {a}^{2}+{b}^{2} \right) }{b\left({a}^{2}-{b}^{2}\right)}}
%     %
%     %
%     \endaligned
%     \]
%     
%     is the geometric locus of $X(80)$.
     

      
          \section{Locus X(80) -- REFLECTION OF INCENTER IN FEUERBACH POINT}
      
      
      The ellipse 
      
      \[ \frac{x^2}{a_{80}^2}+\frac{y^2}{b_{80}^2}=1, \]
      where
      \[   \aligned
      a_{ 80}=&   {\frac { \left( -{b}^{2}+\sqrt {{a}^{4}-{b}^{2}{a}^{2}+{b}^{4}}
      		\right)  \left( {a}^{2}+{b}^{2} \right) }{\left({a}^{2}-{b}^{2}\right) a}}
      %
      %
      \\
      b_{80}=&  {\frac { \left( {a}^{2}-\sqrt {{a}^{4}-{b}^{2}{a}^{2}+{b}^{4}}
      		\right)  \left( {a}^{2}+{b}^{2} \right) }{b\left({a}^{2}-{b}^{2}\right)}}
      %
      %
      \endaligned
      \]
      
      is the geometric locus of $X(80)$.
         \section{Locus X(84) }
      
      
      The ellipse 
      
      \[ \frac{x^2}{a_{84}^2}+\frac{y^2}{b_{84}^2}=1, \]
      where
      \[   \aligned
      a_{ 84}=&   {\frac { \left( {b}^{2}+\sqrt {{a}^{4}-{b}^{2}{a}^{2}+{b}^{4}}
      		\right)  \left( a^2-b^2 \right)    }{{a}^{3}}}
      %
      \\
      b_{84}=&  {\frac { \left( {a}^{2}+\sqrt {{a}^{4}-{b}^{2}{a}^{2}+{b}^{4}}
      		\right)  \left( {a}^{2}-{b}^{2} \right) }{{b}^{3}}}
      %
      %
      \endaligned
      \]
      
      is the geometric locus of $X(84)$.
      
      
      \section{ X(88)- Isogonal Conjugate of $X_{44}$}
      
      X(88) is the billiard.
      
%      The ellipse 
%      
%      \[ \frac{x^2}{a_{88}^2}+\frac{y^2}{b_{88}^2}=1, \]
%      where
%      \[   \aligned
%      a_{ 84}=&   a
%      %
%      \\
%      b_{84}=& b
%      %
%      %
%      \endaligned
%      \]
%      
%      is the geometric locus of $X(88)$.
%      
     \section{Locus X(90) }
     
     
     The ellipse 
     
     \[ \frac{x^2}{a_{90}^2}+\frac{y^2}{b_{90}^2}=1, \]
     where
     \[   \aligned
     a_{ 90}=&   {\frac { \left( {b}^{2} \left( 3\,{a}^{2}-{b}^{2} \right) + \left( {a}
     		^{2}+{b}^{2} \right) \sqrt {{a}^{4}-{b}^{2}{a}^{2}+{b}^{4}} \right) 
     		\left( {a}^{2}-{b}^{2} \right) }{a \left( {a}^{4}+2\,{b}^{2}{a}^{2}-7
     		\,{b}^{4} \right) }}
      \\
     b_{90}=& -{\frac { \left( -{a}^{2} \left( {a}^{2}-3\,{b}^{2} \right) + \left( {
     			a}^{2}+{b}^{2} \right) \sqrt {{a}^{4}-{b}^{2}{a}^{2}+{b}^{4}} \right) 
     		\left( {a}^{2}-{b}^{2} \right) }{b \left( 7\,{a}^{4}-2\,{b}^{2}{a}^{2
     		}-{b}^{4} \right) }}
     %
     \endaligned
     \]
     
     is the geometric locus of $X(90)$.
     
     
     \section{X(100) -- Anticomplement of Feuerbach Point}
     
     It is the billiard
\end{document}

 
 
 \vskip 1cm
 {\author  
 \noindent  Ronaldo A. Garcia\\
 Universidade  Federal de Goi\'as 
\\Instituto de Matem\'atica e Estat\'\i stica  
 \\ Campus Samambaia
\\ Goi\^ania - Goi\'as - Brasil
\\ CEP 74690-900  
\\ ragarcia@ufg.br }
 

 \end{document}
 